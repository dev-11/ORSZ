\documentclass{article}
\usepackage[utf8]{inputenc}
\usepackage{amsmath}
\usepackage{amssymb}
\date{}
\title{ORSZ -- Elovizsgak, beugro kerdesek}
\author{}

\begin{document}
\maketitle

\begin{enumerate}
\item[1] szamolos feladat
\item[2] Ird fel a teljesen diszjunk felbontas definiciojat.
\item[3] Ird fel a detektalas inkrementalis frissitessel megoldoprogramjat.
\item[4] Ird fel a feladat definiciojat megadva azt, hogy mik a realciok es mit jelentenek.
\item[5] Ird fel az invarians es a mindig igaz definiciojat. ($inv_S, true_S$) Vilagits ra egy peldaval a ketto kozotti kulonbsegre.
\item[6] Ird fel a legrovidebb utak megoldoprogramjat egyenlosegi semaban.
\item[7] Ird fel es igazold a H1-es heurisztikat
\item[8] Ird fel egy parhuzamos program es utasitas szuperpoziciojanak definiciojat.
\item[9] Ird fel a megoldas definiciojat! Mit jelent az, hogy $S$ program megfelel a $FP_h$-nak, $Q \in TERM_h$-nak, $P \hookrightarrow_h Q$-nak $K$ invarians mellet (def)?
\item[10] Ird fel az unio es reszhalmazi 1 tetelt.
\item[11] Szamolos feladat
\item[12] Írd fel a teljes diszjunkt felbontés definícióját!
\item[13] Írd fel a detektálás inkrementális frissítéssel megoldóprogramját!
\item[14] Írd fel a feladat definícióját megadva, hogy a definíciók mit jelentenek!
\item[15] Írd fel az invariáns és a mindig igaz definícióját ($inv_S, true_S$) Világíts rá egy példával a kettő közötti különbségre!
\item[16] Írd le a legrövidebb utak (nem szekvenciális) megoldóprogramját!
\item[17] Írd fel és igazold a H2 detektáló heurisztikát!
\item[18] Írd fel egy párhuzamos program és egy utasítás szuperpozíciójának a definícióját!
\item[19] Írd fel a megoldás definícióját! Mit jelent az, hogy $S$ program megfelel a $FP_h$-nak, $Q \in TERM_h$-nak, $P \hookrightarrow_h Q$-nak $K$ invarians mellet (def)?
\item[20] Írd fel lokalitás tételét!
\item[21] $S = (y:=0 \{y:=2x\ ha\  x < 10 \}), R = (x+y<7).\ lf(S,R)$ ?
\item[22] Ird fel az elozo program fixpontjat.
\item[23] Ird fel a fixpontfinomiatas tetelet
\item[24] Mit jelen az, hogy $S$ megfelel a $P \hookrightarrow_S Q$ specifikacios feltetelnek?
\item[25] Ird fel az Unio viselkedesi relaciojat.
\item[26] Ird fel a $\mapsto_S$ definiciojat.
\item[27] Ird fel a szuperpozicio definiciojat.
\item[28] Ird fel a gyenge szuperpozicio definiciojat.
\item[29] Ird fel a variasnfgv tetelt.
\item[30] Ird fel a $P\ detect_S\ Q$ definiciojat
\item[31] $S = (x, y:=1,1 \{y:=2x\ ha\  x < 10 \}), R = (x+y<7).\ lf(S,R)$ ?
\item[32] Ird fel az $inv_S$ definiciojat formalisan.
\item[33] Ird fel mit tudsz a csatornakrol, csatornamulveletekrol, azok szemantikajarol.
\item[34] Ird fel a $P\ detect_S\ Q$ definiciojat
\item[35] Mit jelent az formalisan, hogy $S$ program megfelel a $Q \in TERM_h$ specifikacionak.
\item[36] Ird fel a $\hookrightarrow_S$ definiciojat, formalisan. kifejtve.
\item[37] Ird fel a szuperpozicio definiciojat formalisan.
\item[38] Ird fel a gyenge szuperpozicio definiciojat formalisan.
\item[39] Fejtsd ki mit jelent az ha P stabil
\item[40] Ird fel a fixpontfinomitas tetelet formalisan.
\item[41] szamolos feladat.
\item[42] $\triangleright_S$ es $\mapsto_S$ definicioja
\item[43] Mindig igaz defincioja
\item[44] Def.: $S$ program megfelel $inv_h(P)$-nek
\item[45] Unio es allapotter reszhalmazai.
\item[46] Gyenge szuperpozicio def.
\item[47] Unio viselkedesi relacioja.
\item[48] Legrovidebb utak feladatanak specifikacioja.
\item[49] Legrovidebb utak feladatanak megoldasa egyenlosegi semaban.
\item[50] variansfgv tetele.
\item[51] Szamolos feladat
\item[52] Ird fel a $\hookrightarrow_S$ definiciojat
\item[53] Ird fel az invarians es a mindig igaz definiciojat. ($inv_S, true_S$)
\item[54] Mit jelent az, hogy $S$ program megfelel a $Q \in TERM_h$-nak $K$ invarians mellett?
\item[55] Ird fel a $fixpont_S$ definiciojat.
\item[56] Ird fel a szuperpozicio definiciojat.
\item[57] Ird fel a legrovidebb utak megoldoprogramjat egyenlosegi semaban.
\item[58] Ird fel mit jelent a teljesen diszjunkt felbontas.
\item[59] Ird fel a gyenge szuperpozicio definiciojat.
\item[60] Ird fel az unio viselkedesi relaciojat.
\item[61] Invariáns definíciója
\end{enumerate}
\end{document}
