\documentclass{article}
\usepackage[utf8]{inputenc}
\usepackage{amsmath}
\usepackage{amssymb}
\usepackage{tikz}
\usetikzlibrary{arrows}
\tikzstyle{level 1}=[sibling distance=4cm]
\tikzstyle{level 2}=[sibling distance=1.5cm]
\tikzstyle{level 3}=[sibling distance=.5cm]
\title{Osztott rendszerek szintézise\\3. tétel}
\begin{document}
%\maketitle

\section*{Programkonstrukciók és tulajdonságaik}
A cél valami olyasmi, hogy ha már a komponensekre tudjuk a tulajdonságokat, akkor ezekből könnyedén lehessen a konstrukciókra is számolni, ne kelljen mindent újra.
\subsection*{Unió}
Adottak $S_1$ és $S_2$ programok. Legyen az uniójuk $S_1 \cup S_2$, amit így adhatunk meg:
\begin{align*}
S_1 \cup S_2 = (s_{1,0} \parallel s_{2,0}, S_1 \cup S_2)
\end{align*}
Tehát akkezdeti értékadás a két résztvevő program kezdeti értékadásának párhuzamos futtatása, az értékadáshalmaz pedig a két értékadáshalmaz uniója.

A két program nem szükséges hogy azonos állapottéren fusson, az unió állapottere olyan, aminek altere mindkét résztvevő program állapottere, és ekkor a két programot ki is terjesztjük [az valami olyasmit jelent, hogy mindent megtartunk ami van, az új állapottér-komponeneskre meg minden utasítás skip] erre a közös állapottérre. Ami még nagyon fontos kikötés, és ez az egész unió-dolog csak akkor valósítható meg, ha a két progi eredeti állapotterének legnagyobb közös alterére nézve az $s_{1,0}$ és $s_{2,0}$ kezdeti értékadások ugyanazt adják. Értelmesebben fogalmazva: ha a közös változókat nem próbáljuk két különböző értékre is inicializálni. Ha az egyik kezdeti értékadás rendel egy közös változóhoz valamit, a másik nem, az OK, meg az is OK, ha egyik se rendel semmit. Az nem OK, ha két különböző értéket.

\subsection*{Unió viselkedési relációja}
Legyen $S:=S_1 \cup S_2$. ekkor:
\begin{align*}
\triangleright_s = \triangleright_{S_1} \cap \triangleright_{S_2}
\end{align*}
Ez azért van így, mert a dolog mélyén az lf van. Ott meg minden értékadáshalmazbeli értékadásra kell teljesülnie valaminek, az meg most a két értékadáshalmaz uniója, azaz az összes értékadás. Tehát mindre teljesülnie kell. Ezért van az "éselés", azaz a metszet. Itt amúgy felhasználjuk az lf tulajdonságai tételt is.
\begin{align*}
\mapsto_S = \triangleright_{S_1} \cap \triangleright_{S_2} \cap (\mapsto_{S_1} \cup \mapsto_{S_2})
\end{align*}
Teljesülnie kell a háromszögnek és annak is, hogy akad az unióban olyan utasítás ami átvisz P-ből Q-ba. Nyilván akkor van ilyen ha legalább az egyikben van, ezért van ott unió jel.
\begin{align*}
\hookrightarrow_S = (\triangleright_{S_1} \cap \triangleright_{S_2} \cap (\mapsto_{S_1} \cup \mapsto_{S_2}))^{tdl}
\end{align*}
Ez színtisztán a definíció. A zárójelben ugyanis az egyenesnyíl van.
P-ből Q-ba. Nyilván akkor van ilyen ha legalább az egyikben van, ezért van ott unió jel.
\begin{align*}
\forall Q:sp(s_{1,0}\parallel s_{2,0},Q) \Rightarrow sp(s_{1,0},Q) \land sp(s_{2,0},Q) \rightarrow inv_{S_1}(Q) \land inv_{S_2}(Q) \subseteq inv_S(Q)
\end{align*}
A közös invariánsok, a közös programnak is invariánsai.
\begin{align*}
\varphi_S = \varphi_{S_1} \land \varphi_{S_2}
\end{align*}
A fixpont definíciója is egy nagy éselés, innen jön ez.
\begin{align*}
FP_{S_1} \land FP_{S_2} \subseteq FP_S
\end{align*}
A fixpont tulajdonság olyan ami fixpontban igaz. Az előzőben láttuk, hogy fixpont akkor van, amikor mindkettő programra igaz, hogy fixpontban van. Tehát a közös fixpont-tulajdonságok fixpont-tulajdonságai lesznek a közös programnak.
\begin{align*}
TERM_S = ((\triangleright_{S_1} \cap \triangleright_{S_2} \cap (\mapsto_{S_1} \cup \mapsto_{S_2}))^{tdl})^{(-1)}(\varphi_{S_1} \land \varphi_{S_2})
\end{align*}
A zárójel belsejében az az izé a görbenyíl definíciója (azaz hogy "honnan hova tudok eljutni"), ennek az inverzét ha alkalmazom a fixpontokra, akkor megkapom azt, hogy honnan jutok fixpontba. Na, és hát pont ezek az állapotok, ahonnan terminálok. Amúgy szerintem ez minden programra igaz, nem csak az unióra, persze a görbenyíl és fixpont aktuális definíciójával.
\\\\
Még néhány megjegyzés:
\\\\
Ez: $\triangleright_S = \triangleright_{S_1} \cap \triangleright_{S_2}$, azt jelenti, hogy ha $P\triangleright_sQ$ igaz akkor $P\triangleright_{s_1}Q$ és $P\triangleright_{s_2}Q$ igaz, valamint ez az implikáció fordítva is teljesül.
\\\\
Ha $P\mapsto_{S_1}Q$ és s$P\mapsto_{S_2}Q$ is teljesül, akkor $P\mapsto_SQ$ Fordítva nem igaz! Azért nem igaz, mert egyenesnyílhoz az unió értékadáshalamzából kell, hogy egy értékadás elvigyen $P$-ből $Q$-ba. Ha az mondjuk $S_1$-beli, akkor $S_2$-nek már nem kell ilyen értékadással rendelkeznie, tehát egyáltalán nem biztos, hogy mindkettő programra igaz az egyenes nyíl. Igaz viszont a háromszög, és az egyikre az egyenes nyíl, ezt mutatja a következő összefüggés:
\begin{align*}
(P\mapsto_{S_1}Q \land P\mapsto_{S_2}Q) \rightarrow P \mapsto_S Q
\end{align*}
Nem igaz az, hogy: $(P\hookrightarrow_{S_1}Q \land P\hookrightarrow_{S_2}Q) \rightarrow P \hookrightarrow_S Q$, nézzünk rá egy ellenpéldát ($\hookrightarrow$ és $\leadsto$ ekvivalenciáját fogjuk használni):
\begin{align*}
A &= \underset{x}{\mathbb{Z}}\\
S_1 &= (SKIP, \lbrace x:=x+1 \rbrace)\\
S_2 &= (SKIP, \lbrace x:=x-1 \rbrace)\\
Q &= (|x| > 5)\\
P &= (|x| \leq 5)\\
\end{align*}
$P \hookrightarrow_{S_1} Q$ igaz, hiszen $x$ egyre csak nő, előbb utóbb nagyobb lesz, mint 5. És $P \hookrightarrow_{S_2} Q$ is igaz, hiszen egyre csak csökken, előbb utóbb kisebb lesz, mint -5. De nem igaz, hogy eljutok $P$-ből $Q$-ba $S$-sel, tehát nem igaz, hogy $P\leadsto_S Q$, például az $<s_1, s_2, s_1, s_2, \dots>$ ütemezés mellett, mert ha $x$-et mondjuk 0-ról indítom, akkor 0 és 1 között fog váltakozni végig.\\\\
Tanulság, hogy egyenes nyilat könnyű számolni unióhoz, de görbe nyilat már csak az így megkapott egyenes nyíl tranzitív, diszjunktív lezárásával, ami meg elég melós dolog.
\subsection*{Szuperpozíció}
Behozok új változókat úgy, hogy az eddigi működésen nem változtatok. Egyik lehetőség az, hogy egy új utasítást adok hozzá a halmazhoz, úgy hogy az ne rontsa el az eddigi tulajdonságokat, a másik pedig, hogy az egyik értékadást módosítom úgy, hogy valami új segédváltozónak is értéket adjon szimultán. Mindenesetre az eredeti program változóit maximum lekérdezhetem.
\\
\\
$s_1$ és $s_2$ két azonos állapottéren értelmezett értékadás szuperpozícióján az alábbit értjük, amennyiben
\begin{align*}
VL(s_2) \cap V(s_1) = 0 : s_1 \parallel s_2 ::= \underset{v_i \notin VL(s_2)}{\parallel}(v_i:\in F_{1_i}(v_1,\dots,v_n),\ ha\ \pi_{1_i})\\ \parallel \underset{v_i \in VL(s_2)}{\parallel}(v_i:\in F_{2_i}(v_1,\dots,v_n),\ ha\ \pi_{2_i})
\end{align*}
Azaz, azokhoz az elemekhez, amikhez $s_2$ nem rendel semmit, azokhoz $s_1$ értékadását végezzük el szimultán, amelyikekhez meg igen (és ezekhez feltétel szerint $s_1$ nem), azokhoz meg $s_2$-ét. A $\parallel$ jel a baloldalon a szuperpozíciót jelöli, hisz épp most definiáljuk, a jobb oldalon pedig a szimultán értékadást. És akkor most egy párhuzamos program és egy utasítás szuperpozíciója: Legyen $S = (s_0, {s_1,\dots,s_m})$ egy program az $A$ állapottér egy altere felett, $s$ pedig egy feltételes értékadás $A$ felett, úgy hogy azon változók, amelyeknek $s$ értéket ad $(VL(s))$, nem szerepelnek $S$-ben. Terjesszük ki $S$-t $A$-ra, legyen ez $S'=(s'_0, \lbrace s'_1,\dots, s'_m \rbrace)$. Az alábbi két típusú programot hívjuk szuperpozíciónak (vagy szuperponált programnak):
\begin{enumerate}
\item[a)] $(s'_0, \lbrace s'_1,\dots, s'_m,s \rbrace)$
\item[b)] $(s'_0, \lbrace s'_1,\dots,(s'j \parallel s),\dots, s'_m \rbrace)$
\end{enumerate}
Tehát összesen $m+1$ féle módon tudunk szuperponálni.
\subsection*{Szuperpozíció viselkedési relációja}
Legyen az $A'$ állapottéren adott $S'$ program az $A$ alterén adott $S$ program és az $s$
utasítás egy szuperpozíciója. Jelöljük az $A$ altéren adott $P,Q$ logikai függvények $A'$-re való kiterjesztését $P',Q'$-vel. Ekkor:
\begin{align*}
P \triangleright_S Q &\Rightarrow P' \triangleright_{S'} Q'\\
P \mapsto_S Q &\Rightarrow P' \mapsto_{S'} Q'\\
P \hookrightarrow_S Q &\Rightarrow P' \hookrightarrow_{S'} Q'\\
\forall Q:P \in inv_S(Q) &\Rightarrow P' \in inv_{S'}(Q')\\
R \in FP_S &\Rightarrow R' \in FP_{S'}\\
\varphi_{S'} &= \varphi'_S \land \varphi_S\\
\end{align*}
ahol $\varphi'_S$ a $\varphi_S$ logikai függvény kiterjesztése, a $\varphi_S$ pedig:
\begin{align*}
\varphi_s ::= \underset{i\in [1\dots n]}{\land}(\neg \pi \lor (\pi_{id} \land v_i = F_i(v_1,\dots,v_n)))
\end{align*}
\subsection*{Szekvencia}
????
\end{document}