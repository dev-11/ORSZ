\documentclass{article}
\usepackage[utf8]{inputenc}
\usepackage{amsmath}
\usepackage{amssymb}
\usepackage{tikz}
\usetikzlibrary{arrows}
\renewcommand{\thesubsection}{\alph{subsection})}
\title{Osztott rendszerek szintézise\\9. tétel}
\newcommand{\Claim}{$claim\ $}
\newcommand{\Count}{$count\ $}
\newcommand{\Detect}{$detect_S\ $}
\newcommand{\loopBox}{\overset{n}{\underset{j=1}{\Box}}}
\newcommand{\loopsum}{\overset{n}{\underset{j=1}{\sum}}}
\newcommand{\false}{\downarrow}
\newcommand{\true}{\uparrow}
\newcommand{\pp}{\parallel}
\begin{document}
%\maketitle

\section*{Programok stabil tulajdonságainak megfigyelése}
\subsection{Specifikáció, megoldás szuperpozícióval, heurisztikák a egyes tulajdonságok megfigyelésére, megoldás igazolása heurisztikával}
\subsubsection*{A detektálás és a szuperpozíciós ötlet}
Az "előrejelzésre" a $detect$ tulajdonságot használjuk. Ez a többi tulajdonsághoz hasonlóan programszövegből számolható, az alábbi módon:
\begin{align*}
detect := \lbrace (P \rightarrow Q) \in inv_S(\underset{Q' \in init}{\land} Q') \land Q \hookrightarrow_S P \rbrace
\end{align*}
Tehát ez is egy logikai függvény, azokra az állapotokra igaz, amelyre a fenti két állítás igaz. Célja valami olyasmi, hogy azt mondhassuk vele, hogy ha "$P$ teljesül (ami szintén logikai fv), akkor $Q$ is fog". Tehát az, hogy $P$-ből következik $Q$, az egy invariáns [talpatlan nyíl: implikáció jele]. Mert $P$ előrejelzi $Q$-t. A másik fele pedig arra utal, hogy ha $Q$ véletlenül $P$ előrejelzése nélkül lett igaz, akkor visszalépünk az alaphelyzetbe. A detektálást persze egy új változóval tesszük, ezt hozzá fogjuk szuperponálni a programhoz, a cél az, hogy a mérés ne változtassa meg a program viselkedését.

Példa:
\begin{align*}
S = (s_0, \lbrace x:= x+1 \rbrace)
\end{align*}
Egy új,  $b$ változóval jelezzük előre, ha $x $nagyobb lesz, mint 3, azaz:
\begin{align*}
b\ detects_S(x>3)
\end{align*}
Ehhez alakítsuk át $S$-t, hogy legyen benne $b$

\begin{align*}
S' = (s_0 \pp b:= \false, \lbrace x:= x+1\ \Box\ b:=(x>3) \rbrace)
\end{align*}

Tehát alapvetően úgy működik ez a detect, hogy hozzárakunk egy olyan logikai változót, ami akkor válik igazzá, amikor igaz lesz az az állítás, amit előrejelezni akarunk.

Bizonyítsuk be, hogy $b detect_S (x > 3)$. Ehhez a definícióban adott két állítást kell megnézni.

\begin{enumerate}
\item[1)] $b \rightarrow (x >3)$ invariáns-e?
\begin{enumerate}
\item[a)] kezdetben igaz?\\
$\true \Rightarrow lf(s_0 \pp b:=\false, b \rightarrow x>3) = \false \rightarrow (x >0)^{s_0}$\checkmark (hamisból minden következik)
\item[b)]stabil?\\
$(b \rightarrow x > 3) \Rightarrow lf(S, b \rightarrow x > 3) = lf(x := x+1, b \rightarrow x >3) \land lf(b:=(x>3,b \rightarrow x >3) = (b \rightarrow x >3)^{x\leftarrow x+1} \land (b \rightarrow x>3)^{b \leftarrow b>3}=(\neg b \lor x+1 > 3) \land (x >3 \rightarrow x>3)$ \checkmark
\end{enumerate}
\item[2)] $x>3 \hookrightarrow_S b$ igaz-e?
\begin{enumerate}
\item[a)] Már $x>3 \mapsto_S b$ is igaz, hiszen van $b$-t $x>3$-ra állító utasítás. Ekkor meg a görbe nyíl is igaz lesz def. szerint.
\end{enumerate}
\end{enumerate}
Detektálni nem csak stabil tulajdonságokat lehet!
\subsubsection*{A $count>10$ detektálás problémája}
Az általános alak innentől: $claim\ detect_S\ W$ ahol a $claim$ egy (új, szuperponált) logikai változó, $W$ pedig pedig egy stabil tulajdonság. Jelentse ebben a konkrét példában $W$ az "egy programban a végrehajtott utasítások száma nagyobb mint 10" állítást, ami nyilván stabil (azaz, ha egyszer igaz lett, akkor az is marad\dots). Az utasításszámláló legyen egy új változó, tehát a bizonyítandó állítás ez:
\begin{align*}
claim\ detect_S\ count > 10
\end{align*}
Legyen a szokásos $S=(s_0,\lbrace s_1,\dots,s_n \rbrace)$. Ebben se $claim$ sem a $count$ nincs benne, mindkettőt szuperponáljuk. A $count$-ot minden utasításhoz,hiszen minden végrehajtott utasítás után ezt növelni kéne, ez a szuperponálás egyik lehetséges módja, a $claim$-et pedig a fenti egyszerű példában látott "megszorított unió" módszerrel adjuk a programunkhoz, azaz beiktatunk egy új utasítást, amelyben ő szerepel (ennek a baloldalán tehát csak "új" változó állhat, jobboldalán persze bármi).  Tehát ez most minőségileg két különböző fajta szuperponálás, mind az n értékadáshoz hozzáveszünk egy újabbat, valamint az egész halmazhoz hozzáveszünk egy $n+1$-ediket. Sőt $s_0$-hoz is hozzá kell venni az újak inicializálását.
\begin{align*}
S'=\left( \begin{array}{l}
s_0 \pp count:=0 \pp claim := \false
\left\{ \begin{array}{l}
(\overset{n}{\underset{j=1}{\Box}} s_j \pp count := count +1 ) \Box\\ claim := (count > 10)
\end{array}\right\}
\end{array}\right)
\end{align*}
Bizonyítás
\begin{enumerate}
\item[1)]$(claim \rightarrow count > 10) \in inv_S(\true)$ igaz-e?
\begin{enumerate}
\item[a)]kezdetben igaz, mert a $claim$ hamis, és abból meg bármi következik.\checkmark
\item[b)]stabil is, mert az $n+1$-edik utasítás pont egyenlővé teszi a kettőt, tehát következni is fog, a többi utasítások meg $count$-ot növelik, $claim$-et nem módosítják, tehát ha eddig nagyobb volt, most is az marad. \checkmark
\end{enumerate}
\item[2)]$count > 10 \hookrightarrow_S claim$ igaz-e?
\begin{enumerate}
\item[a)]Egyenes nyílból. A $count>10$ egyszer tutira igaz lesz, és mivel feltétlenül pártatlan az ütemezésünk, ezért a $claim := (count>10)$ is le fog futni. És akkor a $claim$ igaz lesz. De amúgy az egyáltalán nem biztos,
hogy $count = 11$-nél. Lehet, hogy később. \checkmark
\end{enumerate}
\end{enumerate}
Mármost ezzel a megoldással van egy kis probléma. A $count$ mind az $n+1$ utasítás közös változója, ezért a rendszer
nem tud valóban párhuzamos lenni. Ezen lehet javítani, hiszen az "igazi" $S$-től független a két segédváltozónk.
\subsubsection*{Heurisztikák}
Most megvizsgáljuk, milyen heurisztikákkal lehet bizonyos magától értetődő dolgokat bizonyítani, anélkül, hogy a fentieket végigvinnénk minden alkalommal. A heurisztikákat a későbbiekben használni is fogjuk.
\subsubsection*{H1 heurisztika}
Legyen $e$ stabil tulajdonság, amit $x$ logikai változóval szeretnénk előre jelezni. Ha $x$ kezdetben hamis, azaz a kezdő értékadást így szuperponáljuk: $s_0 \pp x:= \false$ és amúgy az $S$-t csak úgy módosítom, hogy hozzá $\Box$-elem az $x$ frissítését, azaz ezt: $x:=e$ akkor kimondhatom további vizsgálat nélkül, hogy $x\ detect_S\ e$.
\subsubsection*{Bizonyítás}
Két dolog kell: $(x \rightarrow e) \in inv_s(\true)$ és $e \hookrightarrow_S x$.
\\
Az invariáns definíció alapján
\begin{enumerate}
\item[1)]$\true \Rightarrow lf(s_0 \pp x:= \false, x\rightarrow e) = (x \rightarrow e)^{x \leftarrow \false, s_0} = (\neg \false \lor s^{s_0})$ (ahol $s_0$ nem változtatja sem $x$-et és $e$-t) Ez igaz, mert a jobb oldalon azonosan igaz van.
\item[2)]$(x \rightarrow e) \Rightarrow lf(S,x \rightarrow e)$ mivan ha $x$ romlik el? Ekkor igazra váltott (hiszen kezdetben hamis volt), mégpedig az $x:=e$ utasítással. Ekkor nyilván $e$ is igaz volt, tehát $\true \rightarrow \true$.
\end{enumerate}
Nézzük a görbenyilat úgy, hogy belátjuk az egyenesnyilat és onnan def szerint van görbe is.
Tehát kell: $e \mapsto_S x$
\begin{enumerate}
\item[1)]$e \triangleright_S x$. Tudjuk, hogy $e \triangleright_S \false$ (stabil). Mivel $\false \Rightarrow x$, ezért jobb oldal gyengítésével kapjuk, hogy $e \triangleright_S x$. \checkmark
\item[2)]$\exists s \in S:e \land \neg x \Rightarrow lf(s,x)$. Nyilván ha van ilyen, akkor az $x:=e$ lesz az. $e \land \neg x \Rightarrow lf(x:=e,x) = (x)^{x\leftarrow e} =e$ \checkmark
\end{enumerate}
\subsubsection*{H2 heurisztika}
Legyen $e$ most monoton növő egész érték (azaz a program utasításai csak növelni tudják), aminek egy bizonyos felső
határon $k$ való túlléptét $x$ egész típusú változóval szeretnénk előre jelezni. Ha $x$ kezdetben $e$ értékét veszi fel, azaz a kezdő értékadást így szuperponáljuk: $s_0 \pp x:=e$ és amúgy az $S$-t csak úgy módosítom, hogy hozzá $\Box$-elem az $x$ frissítését, azaz ezt: $x:=e$ akkor kimondhatom további vizsgálat nélkül, hogy $x$ is monoton növő, valamint hogy $x \geq k\ detect_S\ e \geq k$.
\\\\
Példa: Az előző $S'$ programban ezt szeretnénk a H1-gyel bizonyítani:\\
$claim\ detect_S\ count > 10$

Teljesülnek-e H1 feltételei?
\begin{enumerate}
\item[•] $count>10$ stabil? \checkmark
\item[•] $claim$ logikai változó? \checkmark
\item[•] $claim$ kezdetben hamis $S'$-ben? \checkmark
\item[•] $claim$ csak úgy kap új értéket, hogy $claim:=(count>10)$? \checkmark
\end{enumerate}
Szóval, mivel $e \sim (count>10)$ és $x \sim claim$ szereposztással teljesülnek H1 feltételei, ezért bebizonyíttatott, hogy $claim\ detect_S\ (count>10)$.
\subsubsection*{Bizonyítás}
Két dolog kell: $(x \geq k \rightarrow e \geq k) \in inv_S(\true)$ és $e \geq k \hookrightarrow_S x \geq k$.\\
Invariáns:\\
$\dots = (x < k \lor e \geq k) \in inv_S(\true) = (x < k \lor k \leq e) \in inv_S(\true)$\\
Ha $(x \leq e) \in inv_S(\true)$, akkor nem lehet az, hogy $x$ nagyobb, mint $e$ azaz nem történhet meg a fenti bizonyítandó dolog tagadása. Tehát elég ezt bebizonyítani, és akkor a fenti is kijön. De $x$ kezdetben $e$, tehát kezdetben igaz az egyenlőség. Később pedig a monoton növő $e$-t követi, néha egyenlő vele, néha kisebb.
\\\\
Görbe nyilat pedig megint egyenes nyílból.
\subsubsection*{H3 heurisztika}
A detektálás tranzitív, azaz: 
\begin{align*}
P\ detect_S\ Q \land Q\ detect_S\ R \rightarrow P\ detect_S\ R
\end{align*}
\subsubsection*{Bizonyítás}
Két dologból áll egy ilyen detektálás. Egy görbe nyílból meg egy invariánsból. A görbe nyíl természeténél fogva tranzitív. Míg az invariánsról azt tudom elmondani, hogy ha összeéselek kettőt (márpedig itt összeéselek kettőt), akkor amit kaptam az is invariáns.
\newpage
\subsection{Aszinkronitás biztosítása finomításokkal, heurisztikák alkalmazása a bizonyításhoz}
Az előző példát folytatjuk.
\subsubsection*{Szekvencializálódás elkerülése}
Minden $s_j$ utasításhoz hozzárendelek egy számlálót, ami azt mondja meg, hogy $s_j$ hányszor futott le. 
\begin{align*}
S''=\left( \begin{array}{l}
s_0\pp N:=0 \pp claim:=\false \pp u_1.m:=0 \pp \dots \pp u_n.m:=0\\
\left\{ \begin{array}{l}
\lbrace (\overset{n}{\underset{j=1}{\Box}} s_j \pp u_j.m:=u_j.m+1)\ \Box\\ N:= \sum_{j=1}^{n}u_j.m\ \Box\\ claim:=(N>10)
\end{array}\right\}
\end{array}\right)
\end{align*}
Most az $n$ darab folyamatunk csupa különböző változóval játszik, nem blokkol senki semmit. Kivéve, az $N$-nek értékat adó, amelyik folyamat viszont mindenkit blokkol, de még így is jobb, mint eddig. Most is $claim\ detect_S\ count > 10$ kell a $count$ most a $\overset{n}{\underset{j=1}{\sum}} u_j.m$, ami néha egyenlő $N$-vel is. De csak néha, mert $N$ csak időnként frissül.
\begin{align*}
claim\ detect_S(\sum_{j=1}^{n} u_j.m)>10
\end{align*}
Használjuk H3-at! Ha $claim\ detect_S\ N>10$ (ami H1 miatt oké), és $N>10\ detect_S\ (\overset{n}{\underset{j=1}{\sum}} u_j.m)>10$ akkor a tranzitivitás miatt igaz a bizonyítandó is. $N>10\ detect_S\ (\overset{n}{\underset{j=1}{\sum}} u_j.m)>10$ pedig H2-vel jön ki, $k=11$ szereposztással.
\subsubsection*{Még tutibb megoldás}
Most folyamatonként még egy új változót iktatunk be, mert ezekből sose elég...! Sőt, új detektáló folyamatot is, ami kezeli az új változókat.
\begin{align*}
S''' =\left( \begin{array}{l}
s_0 \pp N:=0 \pp claim:= \false \pp u_1.m,u_1.r:=0,0\pp\dots\pp u_n.m,u_n.r = 0,0,\\
\left\{ \begin{array}{l}
(\loopBox s_j \pp u_j.m:=u_j.m+1)\ \Box\\ (\loopBox u_j.r := u_j.m)\ \Box\\ N:=\loopsum u_j.r\ \Box\\ claim:=(N>10)
\end{array}\right\}
\end{array}\right)
\end{align*}
Tehát mindenkinek van saját számlálója, ezt néha egy \underline{külön} folyamat átmásolja egy másik számlálóba ($u_j.r$), és ezeket fogja majd az $N$ összegezni, ez azért jó, mert így az $u_j.m$-eket nem blokkolja, márpedig azok blokkolása magát az érdemi számítást ($s_j$) is blokkolná. Bizonyítandó megint a $claim\ detect_S  (\overset{n}{\underset{j=1}{\sum}} u_j.m)>10$. És ezt megint tranzitívan fogjuk megtenni: $claim\ detect_S\ N>10$ (H1), $N \geq k\ detect_S\ (\overset{n}{\underset{j=1}{\sum}} u_j.r) \geq k$ (H2) és $(\overset{n}{\underset{j=1}{\sum}} u_j.r) \geq k\ detect_S\ (\overset{n}{\underset{j=1}{\sum}} u_j.m) \geq k$ (ezt nem tudom heurisztikával, de tudom helyette H2-vel egyesével ezeket és ez pont ezt fogja jelenteni: $u_j.r \geq k\ detect_S\ u_j.m \geq k$). És ha ez a három dolog megvan, akkor az első kettőre is nyomok egy H3-at, majd az így kapottra és a harmadikra is nyomok egy H3-at, és meg is van, amit akartam.
\newpage
\subsection{Inkrementális frissítés, gyenge szuperpozíció alkalmazása, megvalósítás tokennel }
\subsubsection*{Inkrementális frissítés}
A cél továbbra is a $claim\ detect_S\ count>10$ biztosítása. Eddig az összes megoldás olyan volt, hogy néha SOK folyamatot blokkolt. Vagy magukat az eredeti folyamatokat, vagy ha később bevezettünk darab segdéd-folyamatot, akkor azokat. Most olyan megoldást adunk, ahol néha csak 1, de legrosszabb esetben is csak 2 processz áll és vár.

Bevezetünk minden $j$-vel jelölt ($j\in [1\dots n]$) folyamathoz egy új változót, az $u_j.delta$-t, ami a $j$-edik utasítás lefutásainak számát jelöli, de nem összesen, mint eddig, hanem csak a legutóbbi frissítés óta. Tehát a delták értékeit rendszeresen átviszem $N$-be és ekkor nyilván nullázom a deltát. Ezért van az, hogy egyszerre csak két folyamat áll, mert a deltákat egyesével adogatom $N$-hez.
\begin{align*}
S''''=\left( \begin{array}{l}
s_0\pp N:=0\pp claim:=\false \pp u_1.delta:=0\pp \dots \pp u_n.delta:=0 \\
\left\{ \begin{array}{l}
(\loopBox s_j \pp u_j.delta := u_j.delta+1)\ \Box\\
(\loopBox N:N=u_j.delta \pp u_j.delta:=0)\ \Box\\
claim:=(N>0)
\end{array}\right\}
\end{array}\right)
\end{align*}
Előnye még az előző megoldással szemben, hogy kevesebb változót használ.\\
Bizonyítása. Az kell, hogy $claim\ detect_S\ count > 10$. De mi itt a $count$? Az $N$ben gyűlik a végrehajtott utasítások száma, de semmi garancia nincs rá, hogy midnen delta lenullázta magát ezért ez az összefüggés érvényes: $count = N + \sum_{j=1}^{n} u_j.delta$. Ezt a programba nem írjuk be, hiszen az pont hogy szekvencializálná azt. A bizonyításhoz viszont használjuk. H1-gyel belátjuk, hogy $claim\ detect_S N>10$, aztán H2-vel szeretnénk, hogy $N \geq k\ detect_S\ count \geq k$. Ebből a kettőből meg H3-mal kijön, hogy $claim\ detect_S\ count>10$.

H1: kell hozzá, hogy $N>10$ stabil (nyilván az, mert $N$ nő), hogy $claim$ logikai változó (az), hogy \Claim kezdetben hamis (az) és hogy \Claim csak úgy változik, hogy $N > 10$ tulajdonság értékét kapja (csak úgy változik). Ezek mind stimmelnek, tehát H1 alkalmazható.

$N \geq k\ detect_S\ count \geq k$ bizonyítása. Na ez mégse megy H2-vel, mert \Count nem egy darab változó.
\\
Úgyhogy def. szerint kell.
\begin{enumerate}
\item[1)] $\forall j \in [1\dots n]: u_j.delta \geq 0$, ez biztos, mondhatni $(u_j.delta \geq 0) \in inv_s(\true)$. Meg az is igaz, hogy $(count = N+ \sum_{j=1}^{n}) \in inv_S(\true)$. Innen (invariánsok éselése invariáns) adódik, hogy: $(count \geq N) \in inv_S(\true)$. Ebből pedig következik, ami kell: $(N \geq k \rightarrow count \geq k) \in inv_S(\true)$.
\item[2)] A másik fele pedig, hogy $(count \geq k \hookrightarrow_S N \geq k)$ szintén igaz, mert $N$ előbb-utóbb frissíti magát és utoléri \Count jelenlegi értékét.
\end{enumerate}
\subsubsection*{Token-alapú megoldás}
Ez egy implementációs módszer egy adott erőforrásra kölcsönös kizárás biztosítására. Körbe jár egy token, és akinél épp van, az dolgozhat, a többi vár. A program \underline{gyenge szuperpozíciót} fog használni. Ennek következik most a definíciója: Adott az $S$ program, annak egy $s_j$ utasításához szuperponáljuk az $r$ utasítást.Így fog kinézni az új $j$-edik utasítás:
\begin{align*}
s'_j = (s_j\ ha\ b \pp r,\ ha\ b)
\end{align*}
De csak akkor, ha az alábbiak teljesülnek:
\begin{enumerate}
\item[•]Igazak a rendes szuperpozíció feltételei (a kiterjesztésben részt vevő feltétel és az új értékadásban levő változók nem macerálják az erdeti $S$-t).
\item[•]Egyszer $b$ igaz lesz. ($\true \hookrightarrow_S b$).
\item[•]$b$ az eredetire nézve stabil ($b\ \triangleright_{S' \setminus \lbrace s' \rbrace} \false$).
\end{enumerate}
Viselkedését tekintve hasonló, mint a sima szuperpozíció, kivéve, hogy az egyenesnyíl tulajdonság a feltétel miatt nem marad meg.De mindenesetre a tokenes megoldáshoz ezt a módszert alkalmazzuk: Legyen $token \in [0\dots n]$ egy új változó, ha a $token$ értéke $j$, akkor a $j$-edik futtathat.
\begin{align*}
S=\left( \begin{array}{l}
s_0\pp N:=0 \pp token:=1 \pp claim:=\false \pp u_1.delta := 0 \pp \dots \pp u_n.delta:=0,\\
\left\{ \begin{array}{l}
(\loopBox s_j,\ ha\ token=j \pp u_j.delta := u_j.delta+1,\ ha\ token = j)\ \Box\\
(\loopBox N:=N+u_j.delta \pp u_j.delta:=0,\ ha token=j)\ \Box\\
claim:=(N+u_j.delta > 10) \pp token:=(token \oplus 1)\ ha token=1
\end{array}\right\}
\end{array}\right)
\end{align*}
Ez tehát más implementáció, mint az eddigiek. Jó kevés pluszváltozó van. Viszont mivel egyszerre csak egy folyamat
dolgozhat, talán valamelyest rosszabb megoldás, mint az inkrementális frissítéses.
\end{document}
