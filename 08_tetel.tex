\documentclass{article}
\usepackage[utf8]{inputenc}
\usepackage{amsmath}
\usepackage{amssymb}
\usepackage{tikz}
\usetikzlibrary{arrows}
\tikzstyle{level 1}=[sibling distance=4cm]
\tikzstyle{level 2}=[sibling distance=1.5cm]
\tikzstyle{level 3}=[sibling distance=.5cm]
\renewcommand{\thesubsection}{\alph{subsection})}
\title{Osztott rendszerek szintézise\\8. tétel}
\begin{document}
%\maketitle

\section*{Elemenkénti feldolgozás}
\subsection{specifikáció, egyszerű megoldás szekvenciális környezetben, teljesen diszjunkt felbontás, kiegyensúlyozott felbontás fogalma}

\subsection{teljesen diszjunkt felbontás párhuzamos előállítása, optimális vágási pontok meghatározása}


\end{document}
