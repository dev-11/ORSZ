ccdr\documentclass{article}
\usepackage[utf8]{inputenc}
\usepackage{amsmath}
\usepackage{amssymb}
\usepackage{tikz}
\usetikzlibrary{positioning}
\usetikzlibrary{arrows}
\title{Osztott rendszerek szintézise\\6. tétel}
\begin{document}
%\maketitle

\section*{Programok leképezése különbözo architektúrákra, programsémák}
A különbségeket a folyamatok kommunikációi jelentik elsősorban.
\subsection*{Architektúrák}
\subsubsection*{Aszinkron, megosztott memóriaarchitektúra}
%\begin{tikzpicture}[scale=2]
%    \tikzstyle{ann} = [draw=none,fill=none,right]
%    \matrix[nodes={draw, ultra thick, fill=blue!20},row sep=0.3cm,column sep=0.5cm] {
%    \node[draw=none,fill=none] {Plain node}; &
%    \node[rectangle] {Rectangle}; &
%    \node[circle] {Circle};\\
%    };
%\end{tikzpicture}
%
%\begin{tikzpicture}[%
%  node distance=1cm,
%  noname/.style={%
%    circle,
%    minimum size=3em,
%    draw
%  }
%]
%  \node[noname] (1)                                             {1};
%  \node[noname] (2) [below=of 1]                                {2};
%  \node[noname] (4) [node distance=1cm and 3mm,below left=of 2] {4};
%  \node[noname] (3) [left=of 4]                                 {3};
%  \node[noname] (5) [below=of 4]                                {5};
%  \node[noname] (6) [node distance=2cm,right=of 5]              {6};
%
%  \path (1) edge                   node {} (2)
%        (2) edge                   node {} (3)
%        (2) edge                   node {} (4)
%        (2) edge                   node {} (6)
%        (3) edge                   node {} (5)
%        (4) edge                   node {} (5)
%        (5) edge [bend right=20pt] node {} (2);
%\end{tikzpicture}

Az $M$ memória közös, a "rekeszeit" sorszámaikkal érjük el. Adottak a $p_i$ processzorok, mindnek külön $c_i$ órajele van. A program legyen $S = ( s_0, \lbrace s_1,\dots s_m \rbrace)$. Minden $S$-beli $s_i$ utasításhoz rendeljünk egy processzort, ami őt futtatja (nyilván egy processzor többet is futtathat): $s_i \rightarrow p_j$. A processzor ezeket ismételgeti naphosszat.

Ha $v_k\ s_i$ baloldali változója (azaz aki értéket kap a folyamatban), (azaz ha $v_k \in LV(s_i)$) akkor az $s_i$-hez rendelt $p_j$ olyan, hogy joga van írni azt az $m_z$ memóriacellát, ahol $v_k$ van.És a jobboldali változókra pedig hasonlóan csak olvasásra.
\subsubsection*{Osztott rendszer}
Itt szintén minden processzorhoz tartozik külön órajel, na de még külön memória is. Egymáséiba nem látnak bele. A kommunikáció itt csatornaváltozókon keresztül valósul meg. Mindenkinek vannak tehát $v \in V$ saját változói és akad néhány $c \in Ch$ csatorna. Esetleg a csatornákhoz van puffer (sor). Két processzor között fizikailag egy csatorna lehet maximum.
\subsubsection*{Szinkron, megosztott rendszer}
Egy, közös órajele van a processzoroknak, a memória is közös. Írhatnak ketten ugyanoda, de csak ugyanazt. Olvasni szabad többnek is egyszerre. Hosszú, vektorszerű értékadások esetén használatos, mely sok-sok változót érint.
\subsection*{Programsémák}
\subsubsection*{Read-only}
Egy program minden utasítása egy processzorhoz tartozik (ugyanakkor egy processzorhoz több utasítás is tartozhat). Egy ilyen program akkor van a read-only sémában, ha minden változójára igaz, hogy legfeljebb egy processzorhoz tartozó utasítások baloldalán fordul elő. (max egy processzor írhasssa)

\begin{itemize}
\item finom szemcsézettség
legfeljebb egy idegen változó legfeljebb egyszer van utasítás jobboldalán (baloldalon nem is lehet, mert readonly, mindenkit csak egy processzor írhat, a többi az "idegen") Példa:
\begin{align*}
s: x&:= x+y+z \parallel y := y+1 \\
s': z&:= \dots
\end{align*}
Itt egy darab idegen változó van (a $z$). Ha pl $s'$-ben $y$ is értéket kap, akkor az már nem jó.
\item durva szemcsézettség
Továbbra is igaz, hogy minden változót legfeljebb egy processzor ír, de most egynél több idegen változó is megengedhető. "Elkérni" nem lehet változót, mert akkor holtpont lesz a blokkolások miatt.
\begin{align*}
s: x&:= x+y+z \parallel y := y+1 \\
s': z&:= \dots \\
s'': y&:= y+1
\end{align*}
\end{itemize}
$\sim$ aszinkron megosztott rendszer architektúra
\subsubsection*{Közös változó séma}

A $v$ változó lokális $p$ processzorra, ha csak a $p$-hez rendelt értékadásokban fordul elő (tehát máshol még olvasásra se). Akkor van ebben a sémában, ha minden értékadásra teljesül, hogy legfeljebb egy nem lokális változó van benne (akár baloldalon, akár jobboldalon… összesen) (max egy processzor nyúlhat hozzá) Ha a fenti két séma kikötései nem teljesülnek, akkor rögzített sorrendben kell az előkerülő nemlokális/idegen változókat lefoglalni, ha el akarjuk kerülni a holtpontot.\\
$\sim$ osztott rendszer architektúra
\subsubsection*{Egyetlen utasítás séma}
Szinkron architektúrán jól kezelhető. A program így néz ki: $S = (s_0, \lbrace \dots, \underset{i}{\parallel} v_i := \dots, \dots \rbrace)$. Tehát mindegyik egy-egy vektorszerű értékadást csinál. Ekkor ha mondjuk ez a programunk: $S = (s_0, \lbrace x:= f(x,y) \parallel y:= g(x,y)\rbrace)$, akkor valamekkora párhuzamosság (aszinkronitás) megengedett (tehát ilyen architektúrán is tud működni (?)), elég csak bizonyos pontokon bevárni egymást pl, amíg nincs meg $f(x,y)$ kiszámítása $x_1$-re, addig még $g$ se foglalkozzon $y_2$-vel és viszont. De ezen kívül már házon belül lerendezhetik az ütemezést.
\subsubsection*{Egyenlőségi séma}
Egyenlőségi séma.\\
funkcionális programozás\\
definiált fv-ek definícióinak sorrendje (?)\\
példa rá 7/a tétel\\
TODO ábrák
\end{document}
